%% Generated by Sphinx.
\def\sphinxdocclass{report}
\documentclass[letterpaper,10pt,english,openany,oneside]{sphinxmanual}
\ifdefined\pdfpxdimen
   \let\sphinxpxdimen\pdfpxdimen\else\newdimen\sphinxpxdimen
\fi \sphinxpxdimen=.75bp\relax

\PassOptionsToPackage{warn}{textcomp}
\usepackage[utf8]{inputenc}
\ifdefined\DeclareUnicodeCharacter
% support both utf8 and utf8x syntaxes
\edef\sphinxdqmaybe{\ifdefined\DeclareUnicodeCharacterAsOptional\string"\fi}
  \DeclareUnicodeCharacter{\sphinxdqmaybe00A0}{\nobreakspace}
  \DeclareUnicodeCharacter{\sphinxdqmaybe2500}{\sphinxunichar{2500}}
  \DeclareUnicodeCharacter{\sphinxdqmaybe2502}{\sphinxunichar{2502}}
  \DeclareUnicodeCharacter{\sphinxdqmaybe2514}{\sphinxunichar{2514}}
  \DeclareUnicodeCharacter{\sphinxdqmaybe251C}{\sphinxunichar{251C}}
  \DeclareUnicodeCharacter{\sphinxdqmaybe2572}{\textbackslash}
\fi
\usepackage{cmap}
\usepackage[T1]{fontenc}
\usepackage{amsmath,amssymb,amstext}
\usepackage{babel}
\usepackage{times}
\usepackage[Bjarne]{fncychap}
\usepackage{sphinx}

\fvset{fontsize=\small}
\usepackage{geometry}

% Include hyperref last.
\usepackage{hyperref}
% Fix anchor placement for figures with captions.
\usepackage{hypcap}% it must be loaded after hyperref.
% Set up styles of URL: it should be placed after hyperref.
\urlstyle{same}
\addto\captionsenglish{\renewcommand{\contentsname}{Contents:}}

\addto\captionsenglish{\renewcommand{\figurename}{Fig.}}
\addto\captionsenglish{\renewcommand{\tablename}{Table}}
\addto\captionsenglish{\renewcommand{\literalblockname}{Listing}}

\addto\captionsenglish{\renewcommand{\literalblockcontinuedname}{continued from previous page}}
\addto\captionsenglish{\renewcommand{\literalblockcontinuesname}{continues on next page}}
\addto\captionsenglish{\renewcommand{\sphinxnonalphabeticalgroupname}{Non-alphabetical}}
\addto\captionsenglish{\renewcommand{\sphinxsymbolsname}{Symbols}}
\addto\captionsenglish{\renewcommand{\sphinxnumbersname}{Numbers}}

\addto\extrasenglish{\def\pageautorefname{page}}

\setcounter{tocdepth}{1}



\title{cnc Documentation}
\date{Apr 30, 2019}
\release{1.0.0}
\author{Stefan Petrovich}
\newcommand{\sphinxlogo}{\vbox{}}
\renewcommand{\releasename}{Release}
\makeindex
\begin{document}

\pagestyle{empty}
\maketitle
\pagestyle{plain}
\sphinxtableofcontents
\pagestyle{normal}
\phantomsection\label{\detokenize{index::doc}}



\chapter{Installation}
\label{\detokenize{installation:installation}}\label{\detokenize{installation::doc}}

\section{Requirements}
\label{\detokenize{installation:requirements}}
This package depends on the following packages, which are automatically
installed with the provided installer:
\begin{enumerate}
\def\theenumi{\arabic{enumi}}
\def\labelenumi{\theenumi .}
\makeatletter\def\p@enumii{\p@enumi \theenumi .}\makeatother
\item {} 
Python 3

\item {} 
numpy

\item {} 
bokeh

\item {} 
tqdm

\end{enumerate}


\section{How To Install}
\label{\detokenize{installation:how-to-install}}
This package is distributed using \sphinxstyleemphasis{setuptools}, and can be installed using the
\sphinxstyleemphasis{setup.py} file provided with the package. The package is installed into the
currently active Python environment using the following command:

\fvset{hllines={, ,}}%
\begin{sphinxVerbatim}[commandchars=\\\{\}]
\PYG{g+gp}{\PYGZgt{}\PYGZgt{}\PYGZgt{} }\PYG{n}{python} \PYG{n}{setup}\PYG{o}{.}\PYG{n}{py} \PYG{n}{install}
\end{sphinxVerbatim}


\chapter{Usage}
\label{\detokenize{usage:usage}}\label{\detokenize{usage::doc}}

\section{CNCOptimizer}
\label{\detokenize{usage:cncoptimizer}}
The \sphinxstyleemphasis{CNCOptimizer} class is the only class needed in order to run the
optimization. An instance of the class gets initialized with the path to the
.code file that needs to be processed, and a flag that tells the optimizer
whether to ignore recipes or not.

After instantiating an \sphinxstyleemphasis{CNCOptimizer} object, the \sphinxstyleemphasis{.optimize} method needs to
be called, after which the \sphinxstyleemphasis{.save} method is used to save the result of the
optimization to a .code file.

A visualization can be generated, if needed, using the \sphinxstyleemphasis{.visualize} method,
which generates a \sphinxstyleemphasis{result.html} file, that can be viewed using any browser.


\section{Full Example}
\label{\detokenize{usage:full-example}}
A full example, which is also provided with this package, in the \sphinxstyleemphasis{run.py}
file, is given below:

\fvset{hllines={, ,}}%
\begin{sphinxVerbatim}[commandchars=\\\{\}]
\PYG{k+kn}{from} \PYG{n+nn}{cnc.optimization} \PYG{k+kn}{import} \PYG{n}{CNCOptimizer}

\PYG{c+c1}{\PYGZsh{} Path to input file}
\PYG{n}{path\PYGZus{}file} \PYG{o}{=} \PYG{l+s+s1}{\PYGZsq{}}\PYG{l+s+s1}{./path\PYGZus{}files/lea\PYGZus{}stpl001\PYGZus{}fused.code}\PYG{l+s+s1}{\PYGZsq{}}

\PYG{c+c1}{\PYGZsh{} Generate optimization object}
\PYG{n}{opt} \PYG{o}{=} \PYG{n}{CNCOptimizer}\PYG{p}{(}\PYG{n}{path\PYGZus{}file}\PYG{p}{,} \PYG{n}{recipe\PYGZus{}grouping}\PYG{o}{=}\PYG{n+nb+bp}{False}\PYG{p}{)}

\PYG{c+c1}{\PYGZsh{} Run the optimization}
\PYG{n}{opt}\PYG{o}{.}\PYG{n}{optimize}\PYG{p}{(}\PYG{p}{)}

\PYG{c+c1}{\PYGZsh{} Write the optimization to a file}
\PYG{n}{opt}\PYG{o}{.}\PYG{n}{save}\PYG{p}{(}\PYG{l+s+s1}{\PYGZsq{}}\PYG{l+s+s1}{result}\PYG{l+s+s1}{\PYGZsq{}}\PYG{p}{)}

\PYG{c+c1}{\PYGZsh{} Generate visualization file}
\PYG{n}{opt}\PYG{o}{.}\PYG{n}{visualize}\PYG{p}{(}\PYG{p}{)}
\end{sphinxVerbatim}


\chapter{Reference}
\label{\detokenize{reference:module-cnc.optimization}}\label{\detokenize{reference:reference}}\label{\detokenize{reference::doc}}\index{cnc.optimization (module)@\spxentry{cnc.optimization}\spxextra{module}}\index{CNCOptimizer (class in cnc.optimization)@\spxentry{CNCOptimizer}\spxextra{class in cnc.optimization}}

\begin{fulllineitems}
\phantomsection\label{\detokenize{reference:cnc.optimization.CNCOptimizer}}\pysiglinewithargsret{\sphinxbfcode{\sphinxupquote{class }}\sphinxcode{\sphinxupquote{cnc.optimization.}}\sphinxbfcode{\sphinxupquote{CNCOptimizer}}}{\emph{file\_path}, \emph{recipe\_grouping=True}}{}
Finds the shortest path for CNC cutting.

Parses the input file, and generates a number of GeneticAlgorithm objects,
for different groups of lines, and performs the optimization in parallel.
Makes visualizations of the cutting trajectory using bokeh.
\subsubsection*{Methods}


\begin{savenotes}\sphinxatlongtablestart\begin{longtable}{\X{1}{2}\X{1}{2}}
\hline

\endfirsthead

\multicolumn{2}{c}%
{\makebox[0pt]{\sphinxtablecontinued{\tablename\ \thetable{} -- continued from previous page}}}\\
\hline

\endhead

\hline
\multicolumn{2}{r}{\makebox[0pt][r]{\sphinxtablecontinued{Continued on next page}}}\\
\endfoot

\endlastfoot

{\hyperref[\detokenize{reference:cnc.optimization.CNCOptimizer.generate_lines_from_file}]{\sphinxcrossref{\sphinxcode{\sphinxupquote{generate\_lines\_from\_file}}}}}()
&
Parses the input file and generates Line objects for every line.
\\
\hline
{\hyperref[\detokenize{reference:cnc.optimization.CNCOptimizer.optimize}]{\sphinxcrossref{\sphinxcode{\sphinxupquote{optimize}}}}}()
&
Groups Line objects, and runs a thread per optimization group.
\\
\hline
{\hyperref[\detokenize{reference:cnc.optimization.CNCOptimizer.save}]{\sphinxcrossref{\sphinxcode{\sphinxupquote{save}}}}}(file\_name)
&
Saves the results of the optimization to a file.
\\
\hline
{\hyperref[\detokenize{reference:cnc.optimization.CNCOptimizer.start_process}]{\sphinxcrossref{\sphinxcode{\sphinxupquote{start\_process}}}}}(all\_optimizations, node\_name, …)
&
Method that creates the optimization object, starts the optimization and saves the result.
\\
\hline
{\hyperref[\detokenize{reference:cnc.optimization.CNCOptimizer.visualize}]{\sphinxcrossref{\sphinxcode{\sphinxupquote{visualize}}}}}()
&
Visualizes the result of the optimization, using the Visualizer class.
\\
\hline
\end{longtable}\sphinxatlongtableend\end{savenotes}
\index{generate\_lines\_from\_file() (cnc.optimization.CNCOptimizer method)@\spxentry{generate\_lines\_from\_file()}\spxextra{cnc.optimization.CNCOptimizer method}}

\begin{fulllineitems}
\phantomsection\label{\detokenize{reference:cnc.optimization.CNCOptimizer.generate_lines_from_file}}\pysiglinewithargsret{\sphinxbfcode{\sphinxupquote{generate\_lines\_from\_file}}}{}{}
Parses the input file and generates Line objects for every line.

\end{fulllineitems}

\index{optimize() (cnc.optimization.CNCOptimizer method)@\spxentry{optimize()}\spxextra{cnc.optimization.CNCOptimizer method}}

\begin{fulllineitems}
\phantomsection\label{\detokenize{reference:cnc.optimization.CNCOptimizer.optimize}}\pysiglinewithargsret{\sphinxbfcode{\sphinxupquote{optimize}}}{}{}
Groups Line objects, and runs a thread per optimization group.

\end{fulllineitems}

\index{save() (cnc.optimization.CNCOptimizer method)@\spxentry{save()}\spxextra{cnc.optimization.CNCOptimizer method}}

\begin{fulllineitems}
\phantomsection\label{\detokenize{reference:cnc.optimization.CNCOptimizer.save}}\pysiglinewithargsret{\sphinxbfcode{\sphinxupquote{save}}}{\emph{file\_name}}{}
Saves the results of the optimization to a file. Adds .code file
extension if it’s not specified.
\begin{quote}\begin{description}
\item[{Parameters}] \leavevmode\begin{description}
\item[{\sphinxstylestrong{file\_name}}] \leavevmode{[}str{]}
Filename of the file which will contain the optimized result.

\end{description}

\end{description}\end{quote}

\end{fulllineitems}

\index{start\_process() (cnc.optimization.CNCOptimizer method)@\spxentry{start\_process()}\spxextra{cnc.optimization.CNCOptimizer method}}

\begin{fulllineitems}
\phantomsection\label{\detokenize{reference:cnc.optimization.CNCOptimizer.start_process}}\pysiglinewithargsret{\sphinxbfcode{\sphinxupquote{start\_process}}}{\emph{all\_optimizations}, \emph{node\_name}, \emph{node}, \emph{pop\_size}, \emph{repro}, \emph{crossover}, \emph{mutation}, \emph{num\_generations}, \emph{progress\_bar\_position}}{}
Method that creates the optimization object, starts the optimization
and saves the result.
\begin{quote}\begin{description}
\item[{Parameters}] \leavevmode\begin{description}
\item[{\sphinxstylestrong{all\_optimizations}}] \leavevmode{[}dict{]}
Dictionary which will be used to store the optimization object that
corresponds to every line group.

\item[{\sphinxstylestrong{node\_name}}] \leavevmode{[}str{]}
The name of the group which is being optimized.

\item[{\sphinxstylestrong{node}}] \leavevmode{[}list of Line{]}
A list of Line objects, which represent the lines that belong to
the group being optimized.

\item[{\sphinxstylestrong{pop\_size}}] \leavevmode{[}int{]}
Size of the population for every generation.

\item[{\sphinxstylestrong{repro}}] \leavevmode{[}float{]}
Probability of reproduction.

\item[{\sphinxstylestrong{crossover}}] \leavevmode{[}float{]}
Probability of crossover.

\item[{\sphinxstylestrong{mutation}}] \leavevmode{[}float{]}
Probability of mutation.

\item[{\sphinxstylestrong{num\_generations}}] \leavevmode{[}int{]}
Number of iterations for which to run the algorithm.

\item[{\sphinxstylestrong{progress\_bar\_position}}] \leavevmode{[}int{]}
Determines the row in which the progress bar will be displayed
while optimizing.

\end{description}

\end{description}\end{quote}

\end{fulllineitems}

\index{visualize() (cnc.optimization.CNCOptimizer method)@\spxentry{visualize()}\spxextra{cnc.optimization.CNCOptimizer method}}

\begin{fulllineitems}
\phantomsection\label{\detokenize{reference:cnc.optimization.CNCOptimizer.visualize}}\pysiglinewithargsret{\sphinxbfcode{\sphinxupquote{visualize}}}{}{}
Visualizes the result of the optimization, using the Visualizer class.

\end{fulllineitems}


\end{fulllineitems}

\index{GeneticAlgorithm (class in cnc.optimization)@\spxentry{GeneticAlgorithm}\spxextra{class in cnc.optimization}}

\begin{fulllineitems}
\phantomsection\label{\detokenize{reference:cnc.optimization.GeneticAlgorithm}}\pysiglinewithargsret{\sphinxbfcode{\sphinxupquote{class }}\sphinxcode{\sphinxupquote{cnc.optimization.}}\sphinxbfcode{\sphinxupquote{GeneticAlgorithm}}}{\emph{nodes}, \emph{pop\_size}, \emph{repro}, \emph{crossover}, \emph{mutation}, \emph{num\_generations}, \emph{progress\_bar\_position=0}}{}
Solves an instance of the travelling salesman problem, for the CNC machine.

Finds the shortest cutting tool travel path (SCTTP) using the genetic
algorithm optimization method.
\subsubsection*{Methods}


\begin{savenotes}\sphinxatlongtablestart\begin{longtable}{\X{1}{2}\X{1}{2}}
\hline

\endfirsthead

\multicolumn{2}{c}%
{\makebox[0pt]{\sphinxtablecontinued{\tablename\ \thetable{} -- continued from previous page}}}\\
\hline

\endhead

\hline
\multicolumn{2}{r}{\makebox[0pt][r]{\sphinxtablecontinued{Continued on next page}}}\\
\endfoot

\endlastfoot

{\hyperref[\detokenize{reference:cnc.optimization.GeneticAlgorithm.crossover}]{\sphinxcrossref{\sphinxcode{\sphinxupquote{crossover}}}}}()
&
Generates part of the population using crossover.
\\
\hline
{\hyperref[\detokenize{reference:cnc.optimization.GeneticAlgorithm.evaluate_generation}]{\sphinxcrossref{\sphinxcode{\sphinxupquote{evaluate\_generation}}}}}()
&
Evaluates the path cost and fitness of the whole generation.
\\
\hline
{\hyperref[\detokenize{reference:cnc.optimization.GeneticAlgorithm.generate_distance_matrix}]{\sphinxcrossref{\sphinxcode{\sphinxupquote{generate\_distance\_matrix}}}}}()
&
Generates matrix of Euclidian distances between every two nodes.
\\
\hline
{\hyperref[\detokenize{reference:cnc.optimization.GeneticAlgorithm.mutation}]{\sphinxcrossref{\sphinxcode{\sphinxupquote{mutation}}}}}()
&
Mutates a set number of individuals in the population, by swapping two genes.
\\
\hline
{\hyperref[\detokenize{reference:cnc.optimization.GeneticAlgorithm.optimize}]{\sphinxcrossref{\sphinxcode{\sphinxupquote{optimize}}}}}()
&
Runs the optimization algorithm trying to find the shortest path.
\\
\hline
{\hyperref[\detokenize{reference:cnc.optimization.GeneticAlgorithm.reproduction}]{\sphinxcrossref{\sphinxcode{\sphinxupquote{reproduction}}}}}()
&
Determines which individuals get to move to the next generation (which ones get cloned).
\\
\hline
\end{longtable}\sphinxatlongtableend\end{savenotes}
\index{crossover() (cnc.optimization.GeneticAlgorithm method)@\spxentry{crossover()}\spxextra{cnc.optimization.GeneticAlgorithm method}}

\begin{fulllineitems}
\phantomsection\label{\detokenize{reference:cnc.optimization.GeneticAlgorithm.crossover}}\pysiglinewithargsret{\sphinxbfcode{\sphinxupquote{crossover}}}{}{}
Generates part of the population using crossover.

Takes two individuals at a time, based on fitness and combines them,
using the Order 1 Crossover method.

\end{fulllineitems}

\index{evaluate\_generation() (cnc.optimization.GeneticAlgorithm method)@\spxentry{evaluate\_generation()}\spxextra{cnc.optimization.GeneticAlgorithm method}}

\begin{fulllineitems}
\phantomsection\label{\detokenize{reference:cnc.optimization.GeneticAlgorithm.evaluate_generation}}\pysiglinewithargsret{\sphinxbfcode{\sphinxupquote{evaluate\_generation}}}{}{}
Evaluates the path cost and fitness of the whole generation.

Path cost is calculated as the Euclidian distance between the second
poin in a node and the first point in the next node. Fitness is
calculated as the maxmimum possible path cost, minus the actual path
cost.

\end{fulllineitems}

\index{generate\_distance\_matrix() (cnc.optimization.GeneticAlgorithm method)@\spxentry{generate\_distance\_matrix()}\spxextra{cnc.optimization.GeneticAlgorithm method}}

\begin{fulllineitems}
\phantomsection\label{\detokenize{reference:cnc.optimization.GeneticAlgorithm.generate_distance_matrix}}\pysiglinewithargsret{\sphinxbfcode{\sphinxupquote{generate\_distance\_matrix}}}{}{}
Generates matrix of Euclidian distances between every two nodes.

\end{fulllineitems}

\index{mutation() (cnc.optimization.GeneticAlgorithm method)@\spxentry{mutation()}\spxextra{cnc.optimization.GeneticAlgorithm method}}

\begin{fulllineitems}
\phantomsection\label{\detokenize{reference:cnc.optimization.GeneticAlgorithm.mutation}}\pysiglinewithargsret{\sphinxbfcode{\sphinxupquote{mutation}}}{}{}
Mutates a set number of individuals in the population, by swapping two
genes.

\end{fulllineitems}

\index{optimize() (cnc.optimization.GeneticAlgorithm method)@\spxentry{optimize()}\spxextra{cnc.optimization.GeneticAlgorithm method}}

\begin{fulllineitems}
\phantomsection\label{\detokenize{reference:cnc.optimization.GeneticAlgorithm.optimize}}\pysiglinewithargsret{\sphinxbfcode{\sphinxupquote{optimize}}}{}{}
Runs the optimization algorithm trying to find the shortest path.

\end{fulllineitems}

\index{reproduction() (cnc.optimization.GeneticAlgorithm method)@\spxentry{reproduction()}\spxextra{cnc.optimization.GeneticAlgorithm method}}

\begin{fulllineitems}
\phantomsection\label{\detokenize{reference:cnc.optimization.GeneticAlgorithm.reproduction}}\pysiglinewithargsret{\sphinxbfcode{\sphinxupquote{reproduction}}}{}{}
Determines which individuals get to move to the next generation (which
ones get cloned).

\end{fulllineitems}


\end{fulllineitems}

\index{Line (class in cnc.optimization)@\spxentry{Line}\spxextra{class in cnc.optimization}}

\begin{fulllineitems}
\phantomsection\label{\detokenize{reference:cnc.optimization.Line}}\pysiglinewithargsret{\sphinxbfcode{\sphinxupquote{class }}\sphinxcode{\sphinxupquote{cnc.optimization.}}\sphinxbfcode{\sphinxupquote{Line}}}{\emph{line\_type}, \emph{starting\_point}, \emph{endpoint}, \emph{recipe}}{}
Line which represents where the CNC head will perform cutting.
\subsubsection*{Methods}


\begin{savenotes}\sphinxatlongtablestart\begin{longtable}{\X{1}{2}\X{1}{2}}
\hline

\endfirsthead

\multicolumn{2}{c}%
{\makebox[0pt]{\sphinxtablecontinued{\tablename\ \thetable{} -- continued from previous page}}}\\
\hline

\endhead

\hline
\multicolumn{2}{r}{\makebox[0pt][r]{\sphinxtablecontinued{Continued on next page}}}\\
\endfoot

\endlastfoot

{\hyperref[\detokenize{reference:cnc.optimization.Line.get_endpoint}]{\sphinxcrossref{\sphinxcode{\sphinxupquote{get\_endpoint}}}}}()
&
Returns the endpoint of a line.
\\
\hline
{\hyperref[\detokenize{reference:cnc.optimization.Line.get_line_type}]{\sphinxcrossref{\sphinxcode{\sphinxupquote{get\_line\_type}}}}}()
&
Returns the type of line.
\\
\hline
{\hyperref[\detokenize{reference:cnc.optimization.Line.get_recipe}]{\sphinxcrossref{\sphinxcode{\sphinxupquote{get\_recipe}}}}}()
&
Returns the recipe of a line.
\\
\hline
{\hyperref[\detokenize{reference:cnc.optimization.Line.get_starting_point}]{\sphinxcrossref{\sphinxcode{\sphinxupquote{get\_starting\_point}}}}}()
&
Returns the starting point of a line.
\\
\hline
{\hyperref[\detokenize{reference:cnc.optimization.Line.get_thikness}]{\sphinxcrossref{\sphinxcode{\sphinxupquote{get\_thikness}}}}}()
&
Returns thikness of EDGEDEL\_LINE line type.
\\
\hline
{\hyperref[\detokenize{reference:cnc.optimization.Line.set_thikness}]{\sphinxcrossref{\sphinxcode{\sphinxupquote{set\_thikness}}}}}(thikness)
&
Set the line thikness, which is only specified for EDGEDEL\_LINE line types.
\\
\hline
\end{longtable}\sphinxatlongtableend\end{savenotes}
\index{get\_endpoint() (cnc.optimization.Line method)@\spxentry{get\_endpoint()}\spxextra{cnc.optimization.Line method}}

\begin{fulllineitems}
\phantomsection\label{\detokenize{reference:cnc.optimization.Line.get_endpoint}}\pysiglinewithargsret{\sphinxbfcode{\sphinxupquote{get\_endpoint}}}{}{}
Returns the endpoint of a line.
\begin{quote}\begin{description}
\item[{Returns}] \leavevmode\begin{description}
\item[{\sphinxstylestrong{endpoint}}] \leavevmode{[}np.array{]}
Numpy array of two coordinates, X2 and Y2, representing the
endpoint of cutting.

\end{description}

\end{description}\end{quote}

\end{fulllineitems}

\index{get\_line\_type() (cnc.optimization.Line method)@\spxentry{get\_line\_type()}\spxextra{cnc.optimization.Line method}}

\begin{fulllineitems}
\phantomsection\label{\detokenize{reference:cnc.optimization.Line.get_line_type}}\pysiglinewithargsret{\sphinxbfcode{\sphinxupquote{get\_line\_type}}}{}{}
Returns the type of line.
\begin{quote}\begin{description}
\item[{Returns}] \leavevmode\begin{description}
\item[{\sphinxstylestrong{line\_type}}] \leavevmode{[}str{]}
Name of line type.

\end{description}

\end{description}\end{quote}

\end{fulllineitems}

\index{get\_recipe() (cnc.optimization.Line method)@\spxentry{get\_recipe()}\spxextra{cnc.optimization.Line method}}

\begin{fulllineitems}
\phantomsection\label{\detokenize{reference:cnc.optimization.Line.get_recipe}}\pysiglinewithargsret{\sphinxbfcode{\sphinxupquote{get\_recipe}}}{}{}
Returns the recipe of a line.
\begin{quote}\begin{description}
\item[{Returns}] \leavevmode\begin{description}
\item[{\sphinxstylestrong{recipe}}] \leavevmode{[}str{]}
Recipe number.

\end{description}

\end{description}\end{quote}

\end{fulllineitems}

\index{get\_starting\_point() (cnc.optimization.Line method)@\spxentry{get\_starting\_point()}\spxextra{cnc.optimization.Line method}}

\begin{fulllineitems}
\phantomsection\label{\detokenize{reference:cnc.optimization.Line.get_starting_point}}\pysiglinewithargsret{\sphinxbfcode{\sphinxupquote{get\_starting\_point}}}{}{}
Returns the starting point of a line.
\begin{quote}\begin{description}
\item[{Returns}] \leavevmode\begin{description}
\item[{\sphinxstylestrong{starting\_point}}] \leavevmode{[}np.array{]}
Numpy array of two coordinates, X1 and Y1, representing the
starting point of cutting.

\end{description}

\end{description}\end{quote}

\end{fulllineitems}

\index{get\_thikness() (cnc.optimization.Line method)@\spxentry{get\_thikness()}\spxextra{cnc.optimization.Line method}}

\begin{fulllineitems}
\phantomsection\label{\detokenize{reference:cnc.optimization.Line.get_thikness}}\pysiglinewithargsret{\sphinxbfcode{\sphinxupquote{get\_thikness}}}{}{}
Returns thikness of EDGEDEL\_LINE line type.
\begin{quote}\begin{description}
\item[{Returns}] \leavevmode\begin{description}
\item[{\sphinxstylestrong{thikness}}] \leavevmode{[}str{]}
Number representing the thikness of the line.

\end{description}

\end{description}\end{quote}

\end{fulllineitems}

\index{set\_thikness() (cnc.optimization.Line method)@\spxentry{set\_thikness()}\spxextra{cnc.optimization.Line method}}

\begin{fulllineitems}
\phantomsection\label{\detokenize{reference:cnc.optimization.Line.set_thikness}}\pysiglinewithargsret{\sphinxbfcode{\sphinxupquote{set\_thikness}}}{\emph{thikness}}{}
Set the line thikness, which is only specified for EDGEDEL\_LINE line
types.
\begin{quote}\begin{description}
\item[{Parameters}] \leavevmode\begin{description}
\item[{\sphinxstylestrong{thikness}}] \leavevmode{[}str{]}
Number representing the thinkess of the line. Not used for
calculations, only when writing to new .code file.

\end{description}

\end{description}\end{quote}

\end{fulllineitems}


\end{fulllineitems}

\phantomsection\label{\detokenize{reference:module-cnc.visualization}}\index{cnc.visualization (module)@\spxentry{cnc.visualization}\spxextra{module}}\index{Visualizer (class in cnc.visualization)@\spxentry{Visualizer}\spxextra{class in cnc.visualization}}

\begin{fulllineitems}
\phantomsection\label{\detokenize{reference:cnc.visualization.Visualizer}}\pysiglinewithargsret{\sphinxbfcode{\sphinxupquote{class }}\sphinxcode{\sphinxupquote{cnc.visualization.}}\sphinxbfcode{\sphinxupquote{Visualizer}}}{\emph{result}, \emph{initial}}{}
Visualizes the result of the optimization for the CNC shortest cutting tool
travel path, using bokeh.
\subsubsection*{Methods}


\begin{savenotes}\sphinxatlongtablestart\begin{longtable}{\X{1}{2}\X{1}{2}}
\hline

\endfirsthead

\multicolumn{2}{c}%
{\makebox[0pt]{\sphinxtablecontinued{\tablename\ \thetable{} -- continued from previous page}}}\\
\hline

\endhead

\hline
\multicolumn{2}{r}{\makebox[0pt][r]{\sphinxtablecontinued{Continued on next page}}}\\
\endfoot

\endlastfoot

{\hyperref[\detokenize{reference:cnc.visualization.Visualizer.generate_tool_path}]{\sphinxcrossref{\sphinxcode{\sphinxupquote{generate\_tool\_path}}}}}(data, step\_size)
&
Generates the whole trajectory of the cutting tool, with a step of \sphinxtitleref{step\_size}.
\\
\hline
{\hyperref[\detokenize{reference:cnc.visualization.Visualizer.populate_plot}]{\sphinxcrossref{\sphinxcode{\sphinxupquote{populate\_plot}}}}}(plot, data)
&
Adds data to the plot, like starting and endpoints of cutting, lines representing the cutting path and lines representnig the non-cutting path.
\\
\hline
{\hyperref[\detokenize{reference:cnc.visualization.Visualizer.split_line}]{\sphinxcrossref{\sphinxcode{\sphinxupquote{split\_line}}}}}(start, end, increment)
&
Function that generates a number of points between two points.
\\
\hline
{\hyperref[\detokenize{reference:cnc.visualization.Visualizer.visualize}]{\sphinxcrossref{\sphinxcode{\sphinxupquote{visualize}}}}}()
&
Generates a plot using bokeh, which displays the initial trajectory and the optimized trajectory of the cutting tool.
\\
\hline
\end{longtable}\sphinxatlongtableend\end{savenotes}
\index{generate\_tool\_path() (cnc.visualization.Visualizer method)@\spxentry{generate\_tool\_path()}\spxextra{cnc.visualization.Visualizer method}}

\begin{fulllineitems}
\phantomsection\label{\detokenize{reference:cnc.visualization.Visualizer.generate_tool_path}}\pysiglinewithargsret{\sphinxbfcode{\sphinxupquote{generate\_tool\_path}}}{\emph{data}, \emph{step\_size}}{}
Generates the whole trajectory of the cutting tool, with a step of
\sphinxtitleref{step\_size}.
\begin{quote}\begin{description}
\item[{Parameters}] \leavevmode\begin{description}
\item[{\sphinxstylestrong{data}}] \leavevmode{[}list of Line{]}
List of line objects, from which the trajectory needs to be
generated.

\item[{\sphinxstylestrong{step\_size}}] \leavevmode{[}int{]}
The Euclidian distance between two steps of the trajectory.

\end{description}

\item[{Returns}] \leavevmode\begin{description}
\item[{\sphinxstylestrong{out}}] \leavevmode{[}touple of np.arrays{]}
Touple where the 1st and 2nd element represents the X and Y
coordinates of the whole cutting tool trajectory, respectively.

\end{description}

\end{description}\end{quote}

\end{fulllineitems}

\index{populate\_plot() (cnc.visualization.Visualizer method)@\spxentry{populate\_plot()}\spxextra{cnc.visualization.Visualizer method}}

\begin{fulllineitems}
\phantomsection\label{\detokenize{reference:cnc.visualization.Visualizer.populate_plot}}\pysiglinewithargsret{\sphinxbfcode{\sphinxupquote{populate\_plot}}}{\emph{plot}, \emph{data}}{}
Adds data to the plot, like starting and endpoints of cutting, lines
representing the cutting path and lines representnig the non-cutting
path.
\begin{quote}\begin{description}
\item[{Parameters}] \leavevmode\begin{description}
\item[{\sphinxstylestrong{plot}}] \leavevmode{[}bokeh figure object{]}
Figure object on which the content of the \sphinxtitleref{data} object will be
displayed.

\item[{\sphinxstylestrong{data}}] \leavevmode{[}list of Line{]}
List of line objects, which have to be drawn on the figure.

\end{description}

\item[{Returns}] \leavevmode\begin{description}
\item[{\sphinxstylestrong{plot}}] \leavevmode{[}bokeh figure object{]}
Figure populated with data from the \sphinxtitleref{data} object.

\end{description}

\end{description}\end{quote}

\end{fulllineitems}

\index{split\_line() (cnc.visualization.Visualizer method)@\spxentry{split\_line()}\spxextra{cnc.visualization.Visualizer method}}

\begin{fulllineitems}
\phantomsection\label{\detokenize{reference:cnc.visualization.Visualizer.split_line}}\pysiglinewithargsret{\sphinxbfcode{\sphinxupquote{split\_line}}}{\emph{start}, \emph{end}, \emph{increment}}{}
Function that generates a number of points between two points.

Generates two vectors, which represent the X and Y coordinates between
points \sphinxtitleref{start} and \sphinxtitleref{end}.
\begin{quote}\begin{description}
\item[{Parameters}] \leavevmode\begin{description}
\item[{\sphinxstylestrong{start}}] \leavevmode{[}np.array{]}
Numpy array containing X and Y of the starting point.

\item[{\sphinxstylestrong{end}}] \leavevmode{[}np.array{]}
Numpy array containing X and Y of the endpoint.

\item[{\sphinxstylestrong{increment}}] \leavevmode{[}int{]}
Euclidian distance between two successive entries in the \sphinxtitleref{start}
and \sphinxtitleref{end} points.

\end{description}

\item[{Returns}] \leavevmode\begin{description}
\item[{\sphinxstylestrong{out}}] \leavevmode{[}touple of np.arrays{]}
Touple where the 1st element is the X coordinates and the 2nd
element is the Y coordinates of the line.

\end{description}

\end{description}\end{quote}

\end{fulllineitems}

\index{visualize() (cnc.visualization.Visualizer method)@\spxentry{visualize()}\spxextra{cnc.visualization.Visualizer method}}

\begin{fulllineitems}
\phantomsection\label{\detokenize{reference:cnc.visualization.Visualizer.visualize}}\pysiglinewithargsret{\sphinxbfcode{\sphinxupquote{visualize}}}{}{}
Generates a plot using bokeh, which displays the initial trajectory and
the optimized trajectory of the cutting tool.

\end{fulllineitems}


\end{fulllineitems}



\renewcommand{\indexname}{Python Module Index}
\begin{sphinxtheindex}
\let\bigletter\sphinxstyleindexlettergroup
\bigletter{c}
\item\relax\sphinxstyleindexentry{cnc.optimization}\sphinxstyleindexpageref{reference:\detokenize{module-cnc.optimization}}
\item\relax\sphinxstyleindexentry{cnc.visualization}\sphinxstyleindexpageref{reference:\detokenize{module-cnc.visualization}}
\end{sphinxtheindex}

\renewcommand{\indexname}{Index}
\printindex
\end{document}